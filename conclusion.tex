\section{Conclusion}
\label{section:conclusion}
If a differential equation is linear and has constant coefficients, then Fourier analysis is especially useful to solve the differential equation by representing all solutions as a linear combination of sines and cosines. Furthermore, it is especially effective to reduce multidimensional, linear PDEs into a series of ODEs due to the simple interpretation of the derivative in Fourier space \citep{howell2016principles}.

With that said, Fourier Analysis is not always the best method for solving differential equations. For example, the Airy ODE equation $y'' - xy = 0$ is not easily solvable using Fourier analysis, due to the non-constant coefficient $x$. Though Fourier analysis can still be used to solve the differential equation by representing the solution in terms of the Fourier Transform of x, the power series approach is a much easier method for solving this differential equation \citep{craig1990linear}.


Furthermore, there are trivial differential equations that Fourier analysis cannot solve due to its requirement needing an absolutely integrable and bounded function. For example, take the differential equation $y' = 1$. It is trivial to find that the solution to this differential equation is $y = x + C$. However, since the integration of function diverges to infinity and the function is unbounded, Fourier Analysis cannot be used to solve this differential equation.


Overall, though the application of Fourier Analysis on differential equations has limitations, where it can be utilized it reduces and solves otherwise difficult-to-solve ordinary differential equations (ODEs) and partial differential equations (PDEs) to a significant extent. Hence, Fourier Analysis is an incredibly powerful and valuable tool for solving differential equations.