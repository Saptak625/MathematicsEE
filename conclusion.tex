\section{Conclusion}
\label{section:conclusion}
Fourier analysis can be used to solve and simplify a great number and variety of ODEs and PDEs and is hence an extremely powerful tool for solving differential equations.
Specifically, if the differential equation is linear and has constant coefficients, then Fourier analysis can be used to solve the differential equation by representing all solutions as a linear combination of sines and cosines.
Furthermore, it can used to reduce multidimensional, linear PDEs into a series of ODEs due to the very easy interpretation of the derivative in Fourier space.

However, it has its limitations and is not always the best method for solving differential equations.
Fourier analysis is not always the best method for solving ODEs with non-constant coefficients.
For example, the Airy equation $y'' - xy = 0$ is not easily solvable using Fourier analysis, due to the non-constant coefficient $x$.
Though not easy, Fourier analysis can still be used to solve the differential equation by representing the solution in terms of the Fourier Transform of x. 
The power series method is a much easier method for solving this differential equation.


However, there are trivial differential equations that Fourier analysis cannot solve due to its requirement needing an absolutely integrable and bounded function. 
For example, take the differential equation $y' = 1$. 
The solution to this differential equation is $y = x + C$, but this function is not absolutely integrable and bounded, so Fourier Analysis cannot be used to solve this differential equation.